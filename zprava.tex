\documentclass[12pt]{article}
\usepackage{epsf,epic,eepic,eepicemu}
%\documentstyle[epsf,epic,eepic,eepicemu]{article}
\usepackage[utf8]{inputenc}

\begin{document}
%\oddsidemargin=-5mm \evensidemargin=-5mm \marginparwidth=.08in
%\marginparsep=.01in \marginparpush=5pt \topmargin=-15mm
%\headheight=12pt \headsep=25pt \footheight=12pt \footskip=30pt
%\textheight=25cm \textwidth=17cm \columnsep=2mm \columnseprule=1pt
%\parindent=15pt\parskip=2pt

\begin{center}
\bf České vysoké učení technické v Praze\\[2mm]
	Fakulta informačních technologií\\[2mm]
	Thákurova 6, 160 00, Praha 6\\[15mm]
	Semestralní projekt MI-PAR 2011/2012:\\
    Paralelní algoritmus \\
    zobecněné Havojské věže\\[5mm]
       Jaroslav Hejral\\
       Jan Langer\\[2mm]
\today
\end{center}

\section{Definice problému a popis sekvenčního algoritmu}
Algoritmus úlohy řeší problém sestavení postupu tahů k vyřešení zoběcněných Hanojských věží pro počet věží větší než 3. Hanojská věž o výšce k je věž k různých žetonů, které jsou uspořádány od nejmenšího k největšímu a rozdíly ve velikostech sousedních žetonů jsou vždy 1. Neuplná hanojská věž o výšce k je věž k různých žetonů, které jsou uspořádány od nejmenších k největším a rozdíly ve velikostech alespoň 1 dvojice sousedních žetonů je alespoň 2. Jeden tah je třesun disku z jedné tyče na jinou s tím, že cílová tyč je buď prázdná nebo je na jeních vrcholu disk s větším průměrem. Úkolem algoritmu je sestavit na cílové tyčce úplnou hanojskou věž s minimálním počtem tahů.

\subsubsection{Vstupní data}

n = přirozené číslo představující celkový počet žetonů, n >= 16. Žeton i, i=1,..,n, má průměr i
s = přirozené číslo představující počet tyček, n/4 >=s > 3
r = číslo cílové tyčky, 1 <= r <= s
V[1,..,s] = množina neúplných hanoiských věží.

Argument V se zadává pomocí skóre jednotlivých věží. Skóre je desítková reprezentace binárního kódu představujícího velikost disků na konkrétní tyči. Skóre tyče se vypočte takto $\displaystyle\sum\limits_{i=0}^n {2^{d(i)-1}}$ kde n=počtu disků na tyči a d(i)=velikosti disku. Součet disků na všech tyčích musí být roven $2^{n-1}$.

Příklad volání programu:
\texttt{./mi-par 5 4 4 1:30:0:0}
%todo obrázek základní konfigurace

\subsection{Výstup algoritmu}
Výstupem algortmu je výpis počtu tahů a posloupnost tahů ve formátu disk, původní tyč -> cílová tyč
%todo příklad výstupu

\subsection{Popis sekvenčního algoritmu}
Algoritmus je typu BB-DFS, tedy prohledávání do hloubky s návratem a mezí. Řešení zadání vždy existuje. Těsná horní mez řešení je vždy dosažitelná a vypočte se podle vzorce $(2^{n/(s-2)-1})(2s-5)$. V případě že je nalezeno řešení je horní mez snížena na jeho velikost mínus jedna (hledáme o jedno menší řešení). V případě kdy zanoření ve stromu stromu řešení překročí aktuálně platnou horní mez dojde k oříznutí neperspektivní větve a provede se návrat. Zde jsme provedli ještě optimalizaci prořezávání a to, že návrat se provede už v případě, že aktuálně zbývající počet kroků do horní meze už nestačí pro přesunutí všech zbývajících disků na cílovou tyč. Touto optimalizací jsme dosáhli oříznutí neperspektivního postupu až o (horní mez-počet disků) dříve. Její významnost se zvyšuje s přibývajícím množstvím tyčí.

V každém cyklu sekvenčního algoritmu je odebrán prvek na vrcholu zásobníku. Je zkontrolováno, zda-li jsme nedošli k řešení, pokud ano je řešení uloženo, sníží se horní mez a smyčka se spustí znovu. Pokud ne, zkontroluje se dosažení meze a zda-li existují možnosti přesunu disků do dalšího kroku. V případě že ano, vygenerují se  a vloží do zásobníku všechny možnosti přesunu disků na tyčích. Smyčka běží dokud není prázdný zásobník. Na konci běhu se vypíše nalezená řešení a počet tahů.

\subsection{Výsledky měření sekvenčního algoritmu na svazku STAR}



\section{Popis paralelního algoritmu a jeho implementace v MPI}

Popište paralelní algoritmus, opět vyjděte ze zadání a přesně
vymezte odchylky, zvláště u algoritmu pro vyvažování zátěže, hledání
dárce, ci ukončení výpočtu.  Popište a vysvětlete strukturu
celkového paralelního algoritmu na úrovni procesů v MPI a strukturu
kódu jednotlivých procesů. Např. jak je naimplementována smyčka pro
činnost procesů v aktivním stavu i v stavu nečinnosti. Jaké jste
zvolili konstanty a parametry pro škálování algoritmu. Struktura a
sémantika příkazové řádky pro spouštění programu.

\section{Naměřené výsledky a vyhodnocení}

\begin{enumerate}
\item Zvolte tři instance problému s takovou velikostí vstupních dat, pro které má
sekvenční algoritmus časovou složitost kolem 5, 10 a 15 minut. Pro
meření čas potřebný na čtení dat z disku a uložení na disk
neuvažujte a zakomentujte ladící tisky, logy, zprávy a výstupy.
\item Měřte paralelní čas při použití $i=2,\cdot,32$ procesorů na sítích Ethernet a InfiniBand.
%\item Pri mereni kazde instance problemu na dany pocet procesoru spoctete pro vas algoritmus dynamicke delby prace celkovy pocet odeslanych zadosti o praci, prumer na 1 procesor a jejich uspesnost.
%\item Mereni pro dany pocet procesoru a instanci problemu provedte 3x a pouzijte prumerne hodnoty.
\item Z naměřených dat sestavte grafy zrychlení $S(n,p)$. Zjistěte, zda a za jakych podmínek
došlo k superlineárnímu zrychlení a pokuste se je zdůvodnit.
\item Vyhodnoďte komunikační složitost dynamického vyvažování zátěže a posuďte
vhodnost vámi implementovaného algoritmu pro hledání dárce a dělení
zásobníku pri řešení vašeho problému. Posuďte efektivnost a
škálovatelnost algoritmu. Popište nedostatky vaší implementace a
navrhněte zlepšení.
\item Empiricky stanovte
granularitu vaší implementace, tj., stupeň paralelismu pro danou
velikost řešeného problému. Stanovte kritéria pro stanovení mezí, za
kterými již není učinné rozkládat výpočet na menší procesy, protože
by komunikační náklady prevážily urychlení paralelním výpočtem.

\end{enumerate}

\section{Závěr}

Celkové zhodnocení semestrální práce a zkušenosti získaných během
semestru.

\section{Literatura}

\appendix

\section{Návod pro vkládání grafů a obrázků do Latexu}

Nejjednodušší způsob vytvoření obrázku je použít vektorový grafický
editor (např. xfig nebo jfig), ze kterého lze exportovat buď
\begin{itemize}
\item postscript formáty (ps nebo eps formát) nebo
\item latex formáty (v pořadí prostý latex, latex s macry epic, eepic, eepicemu). Uvedené pořadí odpovídá růstu
komplikovanosti obrázků který formát podporuje (prostá latex macra
umožnují pouze jednoduché, epic makra něco mezi, je třeba
vyzkoušet).

\end{itemize}
Následující příklady platí pro všechny případy.

Obrázek v postscriptu, vycentrovaný a na celou šířku stránky, s
popisem a číslem. Všimnete si, jak řídit velikost obrazku.
\begin{figure}[ht]
%\epsfysize=3cm \centerline{\epsfbox{VasObrazek.ps}} \caption{Popis
%vašeho obrazku} \label{labelvasehoobrazku}
\end{figure}

Obrázek pouze vložený mezi řádky textu, bez popisu a číslování.\\
\epsfxsize=1cm
%\rule{0pt}{0pt}\hfill\epsfbox{VasObrazek.ps}\hfill\rule{0pt}{0pt}

Latexovské obrázky maji přípony *.latex, *.epic, *.eepic, a
*.eepicemu, respective.
\begin{figure}[ht]
\begin{center}
 %\input VasObrazek.latex
\end{center}
\caption{Popis vašeho obrázku} \label{l1}
\end{figure}
Vypuštením závorek {\tt figure} dostanete opět pouze rámeček v textu
bez čísla a popisu.

Takhle jednoduše můžete poskládat obrázky vedle sebe.
\begin{center}
%\setlength{\unitlength}{0.1mm}\input VasObrazek.epic
\hglue 5mm
%\setlength{\unitlength}{0.15mm}\input VasObrazek.eepic
\hglue 5mm
%\setlength{\unitlength}{0.2mm}\input VasObrazek.eepicemu
\end{center}
Řídit velikost latexovskych obrázků lze příkazem
\begin{verbatim}
\setlength{\unitlength}{0.1mm}
\end{verbatim}
které mění měřítko rastru obrázku, Tyto příkazy je ale současně
nutné vyhodit ze souboru, který xfig vygeneroval.

Pro vytváření grafu lze použít program gnuplot, který umí generovat
postscriptovy soubor, ktery vložíte do Latexu výše uvedeným
způsobem.

\end{document}
